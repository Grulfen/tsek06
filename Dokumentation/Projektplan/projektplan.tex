\documentclass[a4paper,12pt]{article} \usepackage{graphicx}
\usepackage{epstopdf} %\usepackage{gensymb} \usepackage{longtable}
%% Definitioner för LIPS-dokument

\usepackage[english,swedish]{babel}
\usepackage[utf8]{inputenc}
\usepackage[T1]{fontenc}
\usepackage{times}
\usepackage{ifthen}
\usepackage{longtable}

\usepackage[margin=25mm]{geometry}

\usepackage{fancyhdr}
\pagestyle{fancy}
\lhead{}
\chead{\textbf{\LIPSprojekttitel}}
\rhead{\textbf{\textsl{LiTH}}\\\textbf{\LIPSdatum}}
\lfoot{\textbf{\LIPSkursnamn}\\\textbf{\LIPSdokumentansvarig}}
\cfoot{\textbf{\LIPSprojektgrupp}\\\textbf{\LIPSgruppepost}}
\rfoot{\textbf{\textsc{Lip}s}\\\textbf{Sida~\thepage}}

\setlength{\parindent}{0pt}
\setlength{\parskip}{1ex plus 0.5ex minus 0.2ex}


\newcommand{\twodigit}[1]{\ifthenelse{#1<10}{0}{}{#1}}
\newcommand{\dagensdatum}{\number\year-\twodigit{\number\month}-\twodigit{\number\day}}

%% ------------------------------------------
% NYBILD
% Skapar centrerad bild med caption
%   
% #1: Filens url relativt '/bilder/'
% #2:  Caption
% #3: Label
% #4: Skalning
%% ------------------------------------------
\newcommand{\nyBild}[4] 
{\begin{figure}[H]
  \centering
 \includegraphics[angle=0,scale=#4]{bilder/#1}
  \caption{#2}
  \label{fig:#3}
\end{figure}}



%%  Redefinitions of commands containing @
\makeatletter
\makeatother

\newcommand{\LIPStitelsida}{%
{\ }\vspace{45mm}
\begin{center}
  \textbf{\Huge \LIPSdokumenttyp}
\end{center}
\begin{center}
  {\Large Editor: \LIPSredaktor}
\end{center}
\begin{center}
  {\Large \textbf{Version \LIPSversion}}
\end{center}
\vfill
\begin{center}
  {\large Status}\\[1.5ex]
  \begin{tabular}{|*{3}{p{40mm}|}}
    \hline
    Reviewed & \LIPSgranskare & \LIPSgranskatdatum \\
    \hline
    Approved & \LIPSgodkannare & \LIPSgodkantdatum \\
    \hline
  \end{tabular}
\end{center}
\newpage
}


\newenvironment{LIPSprojektidentitet}{%
{\ }\vspace{45mm}
\begin{center}
  {\Large PROJECT IDENTITY}\\[0.5ex]
  {\small
  \LIPSartaltermin, \LIPSprojektgrupp\\
  Linköpings Tekniska Högskola, ISY
  }
\end{center}
\begin{center}
  {\small Group member}\\
%  \begin{tabular}{|p{30mm}|p{40mm}|p{35mm}|p{45mm}|}
  \begin{tabular}{|l|p{45mm}|p{25mm}|l|}
    \hline
    \textbf{Name} & \textbf{Responsibility} & \textbf{Phone} & \textbf{E-mail} \\
    \hline
}%
{%
    \hline
  \end{tabular}
\end{center}
\begin{center}
  {\small
    %\textbf{E-postlista för hela gruppen}: \LIPSgruppepost\\
    %\textbf{Hemsida}: \LIPSgrupphemsida\\[1ex]
    \textbf{Customer}: \LIPSkund\\
    \textbf{Customer Contact}: \LIPSkundkontakt\\
    \textbf{Course Leader}: \LIPSkursansvarig\\
    \textbf{Tutor}: \LIPShandledare\\
  }
\end{center}
\newpage
}
\newcommand{\LIPSgruppmedlem}[4]{\hline {#1} & {#2} & {#3} & {#4} \\}



\newenvironment{LIPSdokumenthistorik}{%
\begin{center}
  Document history\\[1ex]
  \begin{small}
    \begin{tabular}{|l|l|p{60mm}|l|l|}
      \hline
      \textbf{Version} & \textbf{Date} & \textbf{Changes} & \textbf{Edited by} & \textbf{Reviewed} \\
      }%
    {%
      \hline
    \end{tabular}
  \end{small}
\end{center}
}
\newcommand{\LIPSversionsinfo}[5]{\hline {#1} & {#2} & {#3} & {#4} & {#5} \\}

\newcounter{LIPSkravnummer}
\newcounter{LIPSunderkravnummer}[LIPSkravnummer]

\newenvironment{LIPSkravlista}{%
  \begin{longtable}{|p{25mm}|p{85mm}|p{15mm}|}
    }%
  {%
    \hline
  \end{longtable}
}

\newenvironment{LIPSleveranslista}{%
  \begin{tabular}{|p{25mm}|p{20mm}|p{65mm}|p{25mm}|p{5mm}|}
    }%
  {%
    \hline
  \end{tabular}
}


\newcommand{\LIPSkrav}[2]
{\hline
        \stepcounter{LIPSkravnummer}\textbf{Req. nr \arabic{LIPSkravnummer}} &
        %\textbf{{#1}} & 
        {#1} & 
        \textbf{{#2}} 
\\}

\newcommand{\LIPSleverans}[4]
{\hline
        \textbf{{#1}} & 
        {#2} & 
        {#3} & 
        \textbf {{#4}} 
\\}

\newcommand{\LIPSunderkrav}[3]{\hline\stepcounter{LIPSunderkravnummer}\textbf{Requirement nr \arabic{LIPSkravnummer}\Alph{LIPSunderkravnummer}} & \textbf{{#1}} & {#2} & \textbf{{#3}} \\}

%%% Local Variables: 
%%% mode: latex
%%% TeX-master: "kravspec_mall"
%%% End: 



\newcommand{\LIPSartaltermin}{2013/VT}
\newcommand{\LIPSkursnamn}{TSEK06}
\newcommand{\LIPSprojekttitel}{DLL Based Frequency Multiplier}

\newcommand{\LIPSprojektgrupp}{Grupp 7} 
\newcommand{\LIPSgruppepost}{}
\newcommand{\LIPSgrupphemsida}{} 
\newcommand{\LIPSdokumentansvarig}{Gustav Svensk}

\newcommand{\LIPSkund}{ISY, Linköpings universitet, 581\,83 Linköping}
\newcommand{\LIPSkundkontakt}{Amine Ojani} 
\newcommand{\LIPSkursansvarig}{Atila Alvandpour} 
\newcommand{\LIPShandledare}{Amine Ojani}


\newcommand{\LIPSdokumenttyp}{Project Plan} 
\newcommand{\LIPSredaktor}{Nora Björklund} 
\newcommand{\LIPSversion}{0.1} 
\newcommand{\LIPSdatum}{\dagensdatum}

\newcommand{\LIPSgranskare}{} 
\newcommand{\LIPSgranskatdatum}{}
\newcommand{\LIPSgodkannare}{} 
\newcommand{\LIPSgodkantdatum}{}

\begin{document} \LIPStitelsida %% Argument till \LIPSgruppmedlem: namn, roll i
gruppen, telefonnummer, epost \begin{LIPSprojektidentitet}
 
\LIPSgruppmedlem{Nora Björklund}{Project leader}{}{}
\LIPSgruppmedlem{\LIPSdokumentansvarig}{Documentation}{073
6208776}{grulfen3@gmail.com} 
\LIPSgruppmedlem{Christopher Hallberg}{}{0739845945}{chrha007@student.liu.se} 
\LIPSgruppmedlem{Gustaf Bengtz}{}{}{} 
\LIPSgruppmedlem{Johan Berneland}{}{}{}
\end{LIPSprojektidentitet}
\selectlanguage{english}

\tableofcontents{} 
\newpage %% Argument till \LIPSversionsinfo: versionsnummer,
datum, ändringar, utfört av,granskat av 
\addcontentsline{toc}{section}{Document history} \begin{LIPSdokumenthistorik} 
\LIPSversionsinfo{0.1}{}{First draft.}{}{}\end{LIPSdokumenthistorik} 
\newpage


\section{Customer}

\section{Project Outline}

\subsection{Project Goal}

\subsection{Deliveries}

\begin{LIPSleveranslista}
\LIPSleverans{Delivery}{\textbf{Responsible}}{\textbf{Purpose}}{Deadline}\LIPSleverans{}{}{}{}
\hline \end{LIPSleveranslista}

\subsection{Limitations}

\section{Project Phases} 
\subsection{Before Project}
Before starting with the high level design of the DLL the group will hold
preparatory meetings for planning the project and specify the written project
plan. The group members will also study up on relevant material concerning
functionality and implementation of a DLL.
  
\subsection{Under Project}
The project will start with implementing a high level model of the DLL using
the hardware definition language Verilog-A and running simulation on this
model.Thereafter the model will be implemented on gate/transistor level followed
by
simulation. When the DLL behaves correctly on transistor level the work will
continue with the layout of the chip, DRC, parasitic extraction, LVS,
post-layout simulations, modification and chip evaluations. Finally the chip
design will be sent to the foundry for fabrication.

\subsection{After Project}

\section{Organization Plan} 
\subsection{Definition of parties and tasks}
\section{Documentation Plan}

\section{Education}
This section specifies the education of group members and customer respectively.

\subsection{Project Group Education}
Each group member takes responsibility to study material relevant to both the
whole project and his or her designated tasks. 

\subsection{Customer Education}
The information needed for the customer to use the product will be included in
the final report.

\section{Meeting Plan}

\section{Resource Plan} 
\subsection{Persons} 
\subsection{Materiel}
\subsection{Labs}

\section{Milestones and Deadlines} 
\subsection{Milestones}
\subsection{Deadlines}

\section{Time plan}

\section{Quality Plan} 
\subsection{Simulations} 
\subsection{Test plan}
\section{Priorities}

\section{Finishing of Project}

\newpage 
\appendix 
\newpage

\addcontentsline{toc}{section}{References} \begin{thebibliography}{99}
\bibitem{lipskompendiet}\textit{Projektmodellen LIPS - } Svensson, Tomas
\\Studentlitteratur, 2011. \end{thebibliography}

\end{document} 

%%% Local Variables: %%% mode: latex %%% TeX-master: t %%% End: