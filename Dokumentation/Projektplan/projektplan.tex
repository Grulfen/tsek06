\documentclass[a4paper,12pt]{article} \usepackage{graphicx}
\usepackage{epstopdf} %\usepackage{gensymb} \usepackage{longtable}
%% Definitioner för LIPS-dokument

\usepackage[english,swedish]{babel}
\usepackage[utf8]{inputenc}
\usepackage[T1]{fontenc}
\usepackage{times}
\usepackage{ifthen}
\usepackage{longtable}

\usepackage[margin=25mm]{geometry}

\usepackage{fancyhdr}
\pagestyle{fancy}
\lhead{}
\chead{\textbf{\LIPSprojekttitel}}
\rhead{\textbf{\textsl{LiTH}}\\\textbf{\LIPSdatum}}
\lfoot{\textbf{\LIPSkursnamn}\\\textbf{\LIPSdokumentansvarig}}
\cfoot{\textbf{\LIPSprojektgrupp}\\\textbf{\LIPSgruppepost}}
\rfoot{\textbf{\textsc{Lip}s}\\\textbf{Sida~\thepage}}

\setlength{\parindent}{0pt}
\setlength{\parskip}{1ex plus 0.5ex minus 0.2ex}


\newcommand{\twodigit}[1]{\ifthenelse{#1<10}{0}{}{#1}}
\newcommand{\dagensdatum}{\number\year-\twodigit{\number\month}-\twodigit{\number\day}}

%% ------------------------------------------
% NYBILD
% Skapar centrerad bild med caption
%   
% #1: Filens url relativt '/bilder/'
% #2:  Caption
% #3: Label
% #4: Skalning
%% ------------------------------------------
\newcommand{\nyBild}[4] 
{\begin{figure}[H]
  \centering
 \includegraphics[angle=0,scale=#4]{bilder/#1}
  \caption{#2}
  \label{fig:#3}
\end{figure}}



%%  Redefinitions of commands containing @
\makeatletter
\makeatother

\newcommand{\LIPStitelsida}{%
{\ }\vspace{45mm}
\begin{center}
  \textbf{\Huge \LIPSdokumenttyp}
\end{center}
\begin{center}
  {\Large Editor: \LIPSredaktor}
\end{center}
\begin{center}
  {\Large \textbf{Version \LIPSversion}}
\end{center}
\vfill
\begin{center}
  {\large Status}\\[1.5ex]
  \begin{tabular}{|*{3}{p{40mm}|}}
    \hline
    Reviewed & \LIPSgranskare & \LIPSgranskatdatum \\
    \hline
    Approved & \LIPSgodkannare & \LIPSgodkantdatum \\
    \hline
  \end{tabular}
\end{center}
\newpage
}


\newenvironment{LIPSprojektidentitet}{%
{\ }\vspace{45mm}
\begin{center}
  {\Large PROJECT IDENTITY}\\[0.5ex]
  {\small
  \LIPSartaltermin, \LIPSprojektgrupp\\
  Linköpings Tekniska Högskola, ISY
  }
\end{center}
\begin{center}
  {\small Group member}\\
%  \begin{tabular}{|p{30mm}|p{40mm}|p{35mm}|p{45mm}|}
  \begin{tabular}{|l|p{45mm}|p{25mm}|l|}
    \hline
    \textbf{Name} & \textbf{Responsibility} & \textbf{Phone} & \textbf{E-mail} \\
    \hline
}%
{%
    \hline
  \end{tabular}
\end{center}
\begin{center}
  {\small
    %\textbf{E-postlista för hela gruppen}: \LIPSgruppepost\\
    %\textbf{Hemsida}: \LIPSgrupphemsida\\[1ex]
    \textbf{Customer}: \LIPSkund\\
    \textbf{Customer Contact}: \LIPSkundkontakt\\
    \textbf{Course Leader}: \LIPSkursansvarig\\
    \textbf{Tutor}: \LIPShandledare\\
  }
\end{center}
\newpage
}
\newcommand{\LIPSgruppmedlem}[4]{\hline {#1} & {#2} & {#3} & {#4} \\}



\newenvironment{LIPSdokumenthistorik}{%
\begin{center}
  Document history\\[1ex]
  \begin{small}
    \begin{tabular}{|l|l|p{60mm}|l|l|}
      \hline
      \textbf{Version} & \textbf{Date} & \textbf{Changes} & \textbf{Edited by} & \textbf{Reviewed} \\
      }%
    {%
      \hline
    \end{tabular}
  \end{small}
\end{center}
}
\newcommand{\LIPSversionsinfo}[5]{\hline {#1} & {#2} & {#3} & {#4} & {#5} \\}

\newcounter{LIPSkravnummer}
\newcounter{LIPSunderkravnummer}[LIPSkravnummer]

\newenvironment{LIPSkravlista}{%
  \begin{longtable}{|p{25mm}|p{85mm}|p{15mm}|}
    }%
  {%
    \hline
  \end{longtable}
}

\newenvironment{LIPSleveranslista}{%
  \begin{tabular}{|p{25mm}|p{20mm}|p{65mm}|p{25mm}|p{5mm}|}
    }%
  {%
    \hline
  \end{tabular}
}


\newcommand{\LIPSkrav}[2]
{\hline
        \stepcounter{LIPSkravnummer}\textbf{Req. nr \arabic{LIPSkravnummer}} &
        %\textbf{{#1}} & 
        {#1} & 
        \textbf{{#2}} 
\\}

\newcommand{\LIPSleverans}[4]
{\hline
        \textbf{{#1}} & 
        {#2} & 
        {#3} & 
        \textbf {{#4}} 
\\}

\newcommand{\LIPSunderkrav}[3]{\hline\stepcounter{LIPSunderkravnummer}\textbf{Requirement nr \arabic{LIPSkravnummer}\Alph{LIPSunderkravnummer}} & \textbf{{#1}} & {#2} & \textbf{{#3}} \\}

%%% Local Variables: 
%%% mode: latex
%%% TeX-master: "kravspec_mall"
%%% End: 


\newcommand{\degree}{\ensuremath{^\circ}}
\newcommand{\LIPSartaltermin}{2013/VT}
\newcommand{\LIPSkursnamn}{TSEK06}
\newcommand{\LIPSprojekttitel}{DLL Based Frequency Multiplier}

\newcommand{\LIPSprojektgrupp}{Group 7}

\newcommand{\LIPSgruppepost}{}
\newcommand{\LIPSgrupphemsida}{} 
\newcommand{\LIPSdokumentansvarig}{Gustav Svensk}

\newcommand{\LIPSkund}{ISY, Linköpings universitet, 581\,83 Linköping}

\newcommand{\LIPSkundkontakt}{Amin Ojani}
\newcommand{\LIPSkursansvarig}{Atila Alvandpour}
\newcommand{\LIPShandledare}{Amin Ojani}



\newcommand{\LIPSdokumenttyp}{Project Plan} 
\newcommand{\LIPSredaktor}{Nora Björklund} 
\newcommand{\LIPSversion}{0.1} 
\newcommand{\LIPSdatum}{\dagensdatum}

\newcommand{\LIPSgranskare}{} 
\newcommand{\LIPSgranskatdatum}{}
\newcommand{\LIPSgodkannare}{} 
\newcommand{\LIPSgodkantdatum}{}


\begin{document}
\LIPStitelsida

%% Argument till \LIPSgruppmedlem: namn, roll i gruppen, telefonnummer, epost
\selectlanguage{swedish}
\begin{LIPSprojektidentitet}
 
\LIPSgruppmedlem{Nora Björklund}{Project leader}{076 7756
789}{norbj648@student.liu.se}
\LIPSgruppmedlem{\LIPSdokumentansvarig}{Documentation}{073
6208776}{grulfen3@gmail.com} 
\LIPSgruppmedlem{Christopher Hallberg}{}{0739845945}{chrha007@student.liu.se} 
\LIPSgruppmedlem{Gustaf Bengtz}{}{}{} 
\LIPSgruppmedlem{Johan Berneland}{}{}{}
\end{LIPSprojektidentitet}

\selectlanguage{english}

\tableofcontents{} 
\newpage %% Argument till \LIPSversionsinfo: versionsnummer, datum, ändringar,
         %  utfört av,granskat av
\addcontentsline{toc}{section}{Document history} \begin{LIPSdokumenthistorik} 
\LIPSversionsinfo{0.1}{}{First draft.}{}{}\end{LIPSdokumenthistorik} 
\newpage

\section{Customer}
The customer is Amin Ojani from the department of Electrical Engineering.

\section{Project Outline}
This project is the main part of the course TSEK06 - VLSI Chip Design Project.
The purpose of this project is to design a DLL-Based frequency multiplier.
DLL stands for delay-lock loop.

\subsection{Project Goal}
From \cite{project_spec} ``The project goal is to design an integrated circuit
(IC) in complementary metal-oxide
semiconductor (CMOS) technology. Students, participating in this project as
project members
and project leaders, should learn the different steps of the IC design flow.
That includes the
given system architecture analysis, simulation, layout implementation and
verification. The
project students have an optional choice to manufacture the designed IC circuit
on a chip. To
test the manufactured chips, another course (TSEK11) is available after the
project.''


\subsection{Deliverables}

\begin{LIPSleveranslista}
        \LIPSleverans{Delivery}{\textbf{Responsible}}{\textbf{Purpose}}
                     {Deadline}
        \LIPSleverans{High-level simulation result}{Nora Björklund}
                     {Simulation results from high-level, ensuring the functionality of the model}{11/2 -13}
        \LIPSleverans{Transistor-level simulation result}{Nora Björklund}{Simulation results from the gate/transistor level, ensuring the functionality of the design}{11/3 -13}
        \LIPSleverans{Completed chip}{Nora Björklund}{The final design files for the chip}{13/5 -13}
        \LIPSleverans{Final report}{Gustav Svensk}{Report indicating the functionality and how to use the chip}{24/5 -13}
\end{LIPSleveranslista}


\subsection{Limitations}
The final chip shall fulfill the requirements specified in section
\ref{sec:requirements}.
The requirements with priority level High will be focused on first and the other
requirements
will be focused on only after all the requirements with higher priority level
have been met.

\section{System Components}

The system consists of the following components.

\subsection{Phase Detector}

\subsection{Digital Counter}

\subsection{Delay Line}

The delay line delays the signal in steps, the output between the steps will be
combined to produce the output signal of the system. The main components
of the delay line are an even number of inverters.  When the input signal has
passed through the entire delay line it will have been phase-shifted 360\degree. 
The delay of each stage will be set by binary-weighted capacitive loads
controlled by the output from the 6-bit binary counter. Preliminary  the
system will have 8 delay stages, each delaying the signal 45\degree. This makes
it possible to generate an output frequency 4 times that of the input. With this
it is also possible to choose signal components so that the output frequency is
2 times that of the input.

Depending on available space on the chip and available time the preliminary
number of 8 delay stages might be increased further. For example to a total of
16 stages giving 8 times the input frequency on the output. However, since space
is limited this improvement will hold low priority.

\subsection{Phase Combiner}

\section{Project Phases} 
This section briefly describes the outline of the different phases of the
project.
\subsection{Before Project}
Before starting with the high level design of the DLL the group will hold
preparatory meetings for planning the project and specify the written project
plan. The group members will also study up on relevant material concerning
functionality and implementation of a DLL.
  
\subsection{Under Project}
The project will start with implementing a high level model of the DLL using
the hardware definition language Verilog-A and running simulation on this
model.Thereafter the model will be implemented on gate/transistor level followed by
simulation. When the DLL behaves correctly on transistor level the work will
continue with the layout of the chip, DRC, parasitic extraction, LVS,
post-layout simulations, modification and chip evaluations. Finally the chip
layout will be sent to the foundry for fabrication.

\subsection{After Project}
When the chip layout has been sent to the foundry for fabrication a presentation
will be held where the stages of the project and results will be presented. 
Thereafter a more detailed final report will be written and submitted to the 
concerned parties. With this the project is considered finalized. In HT-13 there
is a course where the produced chip can be tested. The majority of the project
group members will take this course at this occasion.


\section{Organization Plan}
This section describes the organization of the project and the different
parties involved.

\subsection{Definition of parties and tasks}
\begin{enumerate}
        \item{Customer: Amin Ojani}
        \item{Project supervisor: Amin Ojani}

                Tasks:
                \begin{itemize}
                        \item{Formulate the project requirements}
                        \item{Provides technical support}
                        \item{Reviews the project documents}
                \end{itemize}
        \item{Project leader}

                Tasks:
                \begin{itemize}
                        \item{Responsible for organization of the team and the project planning}
                        \item{Divides the design and documentation work in an efficient way}
                        \item{Organizes the team meetings as well as the meetings between the team and supervisor}
                        \item{Keeps the supervisor informed about the progress of the project}
                \end{itemize}
        \item{Project design members (including the project leader)}

                Tasks:
                \begin{itemize}
                        \item{Are equally responsible for project planning and design}
                        \item{Participate actively in all the meetings}
                        \item{Support the team and the project leader}
                        \item{Keep the team and project leader informed about the progress of their tasks}
                \end{itemize}
\end{enumerate}



\section{Documentation Plan}

\section{Education}
This section specifies the education of group members and customer respectively.
\subsection{Project Group Education}
Each group member takes responsibility to study material relevant to both the
whole project and his or her designated tasks. 

\subsection{Customer Education}
The information needed for the customer to use the product will be included in
the final report.

\section{Meeting Plan}

\section{Resource Plan} 
This section describes the available resources for the project.
\subsection{Persons} 
Supervisor - Amin Ojani
\subsection{Materiel}
\begin{itemize}
        \item{Scientific publication database}
        \item{IEL - IEEE/IEE Electronic Library, http://www.bibl.liu.se/english/databas/}
        \item{Circuit simulation and layout tool from Cadence$^{\textregistered}$, http://www.cadence.com/}
\end{itemize}

\subsection{Labs}
The project group will have access to the computer 
labs of the department of Electrical Engineering.


\section{Requirements}
\label{sec:requirements}
\begin{LIPSkravlista}
        \LIPSkrav{Design for low power}{Medium}
\LIPSkrav{Integrate as many system components as possible on-chip}{High}       
\LIPSkrav{Schematic and layout must be verified by simulation}{High}
\LIPSkrav{On-chip evaluation should be implemented, for full speed
testing}{High}
\LIPSkrav{Multiplied clock frequency at nominal supply (3.3V)>1 GHz}{High}
\LIPSkrav{Simulated chip power consumption < 100mW (3.3V supply)}{Medium}
\LIPSkrav{Simulated circuit power (normal activity) < 50mW (3.3V
supply)}{Medium}
        \LIPSkrav{Maximum transistor sizing = 20$\mu$m}{Medium}
        \LIPSkrav{Chip core area < 0.27mm$^2$}{High}
        \LIPSkrav{ Total project pin count: 12 }{High}
        \LIPSkrav{ Design technology is AMS 4-Metal 0.35 $\mu$m CMOS }{High}
\LIPSkrav{ The most important system nodes should have off-chip access pins
}{Medium}
        \LIPSkrav{ On-chip current densities < 1 mA/$\mu$m }{High}
        \LIPSkrav{ All requirements fulfilled in “typical”, “slow”, and “fast”
process Medium corners and for temperatures between 25 and 110 $^\circ$C }{High}\end{LIPSkravlista}

\section{Milestones and Deadlines}
This section lists the milestones and deadlines of the project.
The deadlines marked with \textbf{DEADLINE} are hard 
deadlines that can not be adjusted at all.

\begin{LIPSdeadlines}
        \LIPSdeadline{Project selection}{January 18}
        \LIPSdeadline{High-level modeling design and simulation result (report)}{February 11}
        \LIPSdeadline{Gate/Transistor level design and simulation result (report)}{March 11}
        \LIPSdeadline{Layout, DRC, parasitic extraction, LVS, post-layout simulations, modification and chip evaluations}{May 6}
        \LIPSdeadline{\textbf{DEADLINE}, Delivery of the completed chip}{May 13}
        \LIPSdeadline{\textbf{DEADLINE}, Final report and oral presentation}{May 24}
\end{LIPSdeadlines}

\section{Time plan}
The time plan is a seperate document.

\section{Quality Plan} 
\subsection{Simulations} 


\newpage 
\appendix 
\newpage


\addcontentsline{toc}{section}{References}
\begin{thebibliography}{99}
\bibitem{project_spec}\textit{Project: DLL-Based Frequency Multiplier - } Bhide,
Ameya
\\ http://www.ek.isy.liu.se/courses/tsek06/ 2013-01-27
\end{thebibliography}


\end{document} 

%%% Local Variables: %%% mode: latex %%% TeX-master: t %%% End:
