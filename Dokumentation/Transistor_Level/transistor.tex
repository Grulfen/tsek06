\documentclass[a4paper,12pt]{article} \usepackage{graphicx}
\usepackage{epstopdf} %\usepackage{gensymb} \usepackage{longtable}
\usepackage{graphicx}
\usepackage{listings}
\lstset{
language=verilog,
basicstyle=\footnotesize,
breaklines=true
}

%% Definitioner för LIPS-dokument

\usepackage[english,swedish]{babel}
\usepackage[utf8]{inputenc}
\usepackage[T1]{fontenc}
\usepackage{times}
\usepackage{ifthen}
\usepackage{longtable}

\usepackage[margin=25mm]{geometry}

\usepackage{fancyhdr}
\pagestyle{fancy}
\lhead{}
\chead{\textbf{\LIPSprojekttitel}}
\rhead{\textbf{\textsl{LiTH}}\\\textbf{\LIPSdatum}}
\lfoot{\textbf{\LIPSkursnamn}\\\textbf{\LIPSdokumentansvarig}}
\cfoot{\textbf{\LIPSprojektgrupp}\\\textbf{\LIPSgruppepost}}
\rfoot{\textbf{\textsc{Lip}s}\\\textbf{Sida~\thepage}}

\setlength{\parindent}{0pt}
\setlength{\parskip}{1ex plus 0.5ex minus 0.2ex}


\newcommand{\twodigit}[1]{\ifthenelse{#1<10}{0}{}{#1}}
\newcommand{\dagensdatum}{\number\year-\twodigit{\number\month}-\twodigit{\number\day}}

%% ------------------------------------------
% NYBILD
% Skapar centrerad bild med caption
%   
% #1: Filens url relativt '/bilder/'
% #2:  Caption
% #3: Label
% #4: Skalning
%% ------------------------------------------
\newcommand{\nyBild}[4] 
{\begin{figure}[H]
  \centering
 \includegraphics[angle=0,scale=#4]{bilder/#1}
  \caption{#2}
  \label{fig:#3}
\end{figure}}



%%  Redefinitions of commands containing @
\makeatletter
\makeatother

\newcommand{\LIPStitelsida}{%
{\ }\vspace{45mm}
\begin{center}
  \textbf{\Huge \LIPSdokumenttyp}
\end{center}
\begin{center}
  {\Large Editor: \LIPSredaktor}
\end{center}
\begin{center}
  {\Large \textbf{Version \LIPSversion}}
\end{center}
\vfill
\begin{center}
  {\large Status}\\[1.5ex]
  \begin{tabular}{|*{3}{p{40mm}|}}
    \hline
    Reviewed & \LIPSgranskare & \LIPSgranskatdatum \\
    \hline
    Approved & \LIPSgodkannare & \LIPSgodkantdatum \\
    \hline
  \end{tabular}
\end{center}
\newpage
}


\newenvironment{LIPSprojektidentitet}{%
{\ }\vspace{45mm}
\begin{center}
  {\Large PROJECT IDENTITY}\\[0.5ex]
  {\small
  \LIPSartaltermin, \LIPSprojektgrupp\\
  Linköpings Tekniska Högskola, ISY
  }
\end{center}
\begin{center}
  {\small Group member}\\
%  \begin{tabular}{|p{30mm}|p{40mm}|p{35mm}|p{45mm}|}
  \begin{tabular}{|l|p{45mm}|p{25mm}|l|}
    \hline
    \textbf{Name} & \textbf{Responsibility} & \textbf{Phone} & \textbf{E-mail} \\
    \hline
}%
{%
    \hline
  \end{tabular}
\end{center}
\begin{center}
  {\small
    %\textbf{E-postlista för hela gruppen}: \LIPSgruppepost\\
    %\textbf{Hemsida}: \LIPSgrupphemsida\\[1ex]
    \textbf{Customer}: \LIPSkund\\
    \textbf{Customer Contact}: \LIPSkundkontakt\\
    \textbf{Course Leader}: \LIPSkursansvarig\\
    \textbf{Tutor}: \LIPShandledare\\
  }
\end{center}
\newpage
}
\newcommand{\LIPSgruppmedlem}[4]{\hline {#1} & {#2} & {#3} & {#4} \\}



\newenvironment{LIPSdokumenthistorik}{%
\begin{center}
  Document history\\[1ex]
  \begin{small}
    \begin{tabular}{|l|l|p{60mm}|l|l|}
      \hline
      \textbf{Version} & \textbf{Date} & \textbf{Changes} & \textbf{Edited by} & \textbf{Reviewed} \\
      }%
    {%
      \hline
    \end{tabular}
  \end{small}
\end{center}
}
\newcommand{\LIPSversionsinfo}[5]{\hline {#1} & {#2} & {#3} & {#4} & {#5} \\}

\newcounter{LIPSkravnummer}
\newcounter{LIPSunderkravnummer}[LIPSkravnummer]

\newenvironment{LIPSkravlista}{%
  \begin{longtable}{|p{25mm}|p{85mm}|p{15mm}|}
    }%
  {%
    \hline
  \end{longtable}
}

\newenvironment{LIPSleveranslista}{%
  \begin{tabular}{|p{25mm}|p{20mm}|p{65mm}|p{25mm}|p{5mm}|}
    }%
  {%
    \hline
  \end{tabular}
}


\newcommand{\LIPSkrav}[2]
{\hline
        \stepcounter{LIPSkravnummer}\textbf{Req. nr \arabic{LIPSkravnummer}} &
        %\textbf{{#1}} & 
        {#1} & 
        \textbf{{#2}} 
\\}

\newcommand{\LIPSleverans}[4]
{\hline
        \textbf{{#1}} & 
        {#2} & 
        {#3} & 
        \textbf {{#4}} 
\\}

\newcommand{\LIPSunderkrav}[3]{\hline\stepcounter{LIPSunderkravnummer}\textbf{Requirement nr \arabic{LIPSkravnummer}\Alph{LIPSunderkravnummer}} & \textbf{{#1}} & {#2} & \textbf{{#3}} \\}

%%% Local Variables: 
%%% mode: latex
%%% TeX-master: "kravspec_mall"
%%% End: 


\newcommand{\degree}{\ensuremath{^\circ}}
\newcommand{\LIPSartaltermin}{2013/VT}
\newcommand{\LIPSkursnamn}{TSEK06}
\newcommand{\LIPSprojekttitel}{DLL Based Frequency Multiplier}

\newcommand{\LIPSprojektgrupp}{Group 7}

\newcommand{\LIPSgruppepost}{}
\newcommand{\LIPSgrupphemsida}{} 
\newcommand{\LIPSdokumentansvarig}{Gustav Svensk}

\newcommand{\LIPSkund}{ISY, Linköpings universitet, 581\,83 Linköping}

\newcommand{\LIPSkundkontakt}{Amin Ojani}
\newcommand{\LIPSkursansvarig}{Atila Alvandpour}
\newcommand{\LIPShandledare}{Amin Ojani}

\newcommand{\LIPSdokumenttyp}{Transistor Level Simulation} 
\newcommand{\LIPSredaktor}{Nora Björklund} 
\newcommand{\LIPSversion}{1.0} 
\newcommand{\LIPSdatum}{\dagensdatum}

\newcommand{\LIPSgranskare}{} 
\newcommand{\LIPSgranskatdatum}{}
\newcommand{\LIPSgodkannare}{} 
\newcommand{\LIPSgodkantdatum}{}

\begin{document}
\LIPStitelsida

%% Argument till \LIPSgruppmedlem: namn, roll i gruppen, telefonnummer, epost
\selectlanguage{swedish}
\begin{LIPSprojektidentitet}
 
\LIPSgruppmedlem{Nora Björklund}{Project leader}{076 7756
789}{norbj648@student.liu.se}
\LIPSgruppmedlem{\LIPSdokumentansvarig}{Documentation}{073
6208776}{grulfen3@gmail.com} 
\LIPSgruppmedlem{Christopher Hallberg}{}{0739845945}{chrha007@student.liu.se} 
\LIPSgruppmedlem{Gustaf Bengtz}{}{0707367307}{gbengtz@gmail.com} 
\LIPSgruppmedlem{Johan Berneland}{}{0704988329}{johbe915@student.liu.se}
\end{LIPSprojektidentitet}


\selectlanguage{english}

\tableofcontents{} 
\newpage %% Argument till \LIPSversionsinfo: versionsnummer, datum, Ändringar,
         %  utfört av,granskat av
\addcontentsline{toc}{section}{Document history}
\begin{LIPSdokumenthistorik} 
        \LIPSversionsinfo{0.1}{}{First draft.}{}{}
\end{LIPSdokumenthistorik} 
\newpage


\section{Overview}
An overview of the system can be seen in figure\ref{fig:system}.
\begin{figure}[h]
        \centering
        \includegraphics[width=0.7\textwidth]{../Bilder/DIA_high_level.png}
        \caption{Overview of the system}
        \label{fig:system}
\end{figure}
\section{Block Level and Description}
\subsection{Phase Detector}
The Phase Detector (PD) (Figure \ref{fig:PD}) is a D-flipflop (DFF) (Figure \ref{fig:DFF}) without reset and inverted output signal.

\begin{figure}[ht]
\centering
\includegraphics[width=0.7\textwidth, angle = 270]{../Bilder/Phase_detector_trans.png}
\caption{The Phase Detector}
\label{fig:PD}
\end{figure}

Noteworthy is that the inverter after the first transmission gate has got counterintuitive sizing of the PMOS and NMOS transistors.
Using a NMOS being almost five times wider than the PMOS gives better rise and fall time. Except that, the NMOS are 1/3 the size of the
PMOS, with a fanout of 3. The transistors on the input side were set to being as small as possible, and the size increases towards the
output, to buff the signal before sending it to the next subsystem.

\begin{figure}[ht]
\centering
\includegraphics[width=0.7\textwidth, angle = 270]{../Bilder/DFF-1.png}
\caption{D-flipflop}
\label{fig:DFF}
\end{figure}

The sizing of the transistors in the DFF have been found by sweeping on the variables that were interresting.

The set-up time for the PD, which can be seen in Figure \ref{fig:set_up_PD}, is ... ps.

\begin{figure}[ht]
\centering
\includegraphics[width=0.7\textwidth, angle = 270]{../Bilder/set_up_PD.png}
\caption{Set-up time for PD}
\label{fig:set_up_PD}
\end{figure}

The time it takes for the DFF to propagate the signal from the input to the output is, as can be seen in Figure \ref{fig:propagate_DFF},
takes ... ps.

\begin{figure}[ht]
\centering
\includegraphics[width = 0.7\textwidth, angle = 270]{../Bilder/propagate_DFF.png}
\caption{Propagation delay in DFF}
\label{fig:propagate_DFF}
\end{figure}

\subsection{Counter}
\subsection{Digitally Controlled Delay Line}
\subsection{Lock Detector}
\subsection{Frequency Multiplier}
\section{Simulation Result}
\subsection{Whole System Simulation}
\subsection{Phase Detector}
\subsection{Counter}
\subsection{Digitally Controlled Delay Line}
\subsection{Lock Detector}
\subsection{Frequency Multiplier}
\section{Risks and Delays}
\subsection{Counter}
\subsection{Digitally Controlled Delay Line}
\subsection{Frequency Multiplier}
\newpage 
\appendix 
\newpage

\addcontentsline{toc}{section}{References}
\begin{thebibliography}{99}
\bibitem{dll_clock}\textit{A Low-Power Digital DLL-Based Clock Generator in Open-Loop Mode - }
Behzad Mesgarzadeh, Atila Alvandpour \\
IEEE Journal of Solid-State Circuits. Vol. 44. No. 7. July 2009
\bibitem{lock_detect}\textit{A 62.5-625-Mhz Anti-Reset All-Digital Delay-Locked Loop - }
Shao-Ku Kao, Bo-Jiun Chen and Shen-Iuan Liu \\
IEEE Transations on Circuits and Systems - II: Express Briefs, Vol. 54 No. 7, July 2007

\end{thebibliography}
\end{document} 
%%% Local Variables: %%% mode: latex %%% TeX-master: t %%% End:
